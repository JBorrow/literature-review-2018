A quick comparison of relevant models used in \citet{dubois_black_2015}
(the \hagn{} paper, henceforth referred to as D15) and
\citet{angles-alcazar_black_2017} (the relevant \fire paper, henceforth
referred to as A17) is presented in Table \ref{tab:comparison}. It is important
to note that in the main \fire{} paper \citep{hopkins_fire-2_2017} and 
calculations there is no black hole physics implemented, and only in A17 are
the effects of black holes considered.

As well as the sub-grid physics models implemented in both of the simulation
suites, there are some major differences in the numerics between the two
projects. Below these differences are considered individually.

\subsection{Hydrodynamics}

\hagn{} and \fire{} choose fundamentally different paradigms for their
hydrodynamical calculations. \fire{} uses the `GIZMO' code in MFM mode
(a meshless finite mass method, see \citet{hopkins_gizmo_2015}), whereas
\hagn{} uses the `RAMSES' code, using an AMR (adaptive mesh refinment, see
\citet{romain_cosmological_2001}) technique. The major difference between
these two methods is that the MFM code uses \emph{particles} to represent
the fluids in the simulations, and the AMR code uses \emph{cells} defined
on a grid structure.

The major benefit to using a particle-based simulation code is that it is
automatically adaptive; regions that have a higher density have a higher
resolution simply by the nature of there being more particles there. An AMR
code, on the other hand, requires a refinment algorithm which refines the mesh
typically when a cell breaches some density threshold. Historically, AMR codes
have been prized for their more accurate (typically second-order)
hydrodynamical calculations, but the GIZMO code can take advantage of a
meshless structure (as opposed to the usual smoothed particle hydrodynamics
used for particle-based simulations, see e.g.
\citet{springel_cosmological_2005}) to solve equations to an equally high order
when paired with an appropriate Riemann solver \citep{hopkins_gizmo_2015,
hopkins_new_2017}.

Along with the underlying methods used to calculate hydrodynamical
acclerations, there is also the question of resolution. ?????????????

\subsection{Gravity}

% We want our table to span across multiple columns!
\end{multicols}

% Change the distance between the lines to make it more readable
\renewcommand{\arraystretch}{2}

\begin{table}
  \centering
    \begin{tabularx}{\textwidth}{cXX}
    Model & \hagn & \fire \\
    \hline
    Star Formation & Schmidt Law: $\dot{\rho}_* = \epsilon_* \rho / t_{ff}$ at high densities ($\rho_0 > 250$ $n_H$ cm$^{-3}$). Resolution dependent. & Contains resolved molecular clouds. SF is activated when gas is self-gravitating, self-shielded, and has a density $\rho_0 > 1000$ $n_H$ cm$^{-3}$. SF in these regions follows a Schmidt Law: $\dot{\rho}_* = \rho_{mol} / t_{ff}$ where $\rho_{mol}$ is the density of molecular gas. \\
    Stellar Feedback & Gas pressure is enhanced to follow $T = T_0 (\rho/\rho_0$ in the cells surrounding the SF event. & Stellar populations (SP) are individually evolved (\textsc{Starburst99}) and use local radiative transfer based on the SP properties. Winds are also found directly from the SP. \\
    Supernovae & Supernovae energy of $e_{SN} = 10^{50}$ erg M$_{\odot}^{-1}$ is injected to neighboring cells using the Sedov-Taylor exact solution including mass, momentum and energy. & A voronoi-like mesh is constructed around the star particle, and energy, momentum, and mass are deposited in a weighted manner on the faces using the exact Sedov-Taylor solution \citep[see][]{hopkins_how_2017}.\\
    Black Hole Growth & Uses a \citet{bondi__1957} solution such that $\dot{M}_{BH} \propto M_{BH}^2 \rho/(c^2 + u^2)^{3/2}$, and includes terms for black hole spin to cap accretion at the \citet{eddington__xxxx} limit. & Uses the \citet{hopkins_x_2011} `Torque' formalism such that $\dot{M}_{BH} \propto f_d^{3/2} M_d R_0^{-3/2} M_{BH}^{1/6}$. The \citet{eddington__xxxx} limit can be exceeded by 10 times in this model. \\
    Black Hole Dynamics & A drag force is added to emulate dynamical friction. & The black hole is given a dynamical mass of $M_{dyn} = 60M_{BH}$ initially. This limit is removed once $M_{BH} = M_{dyn}$. \\
    AGN Feedback & Two modes: radio ($\dot{M}_{BH}/\dot{M}_{edd} < 0.01$) where jets are ejected at $v = 10^{4}$ km s$^{-1}$, and quasar, where energy is deposited isotropically to neighboring cells such that $\dot{E}_{AGN} \propto \dot{M}_{BH}$ is calibrated to reproduce the $M_{BH}$-$M_b$ and $M_{BH}$-$\sigma_b$ relations. & None. \\
    Cooling & H, He and metals along with a UV background after reionization at $z=10$ is included in a \citet{sutherland_cooling_1993} model between $10^4$ and $10^{8.5}$ K. & A \citet{hopkins_x_2014} model is used with 11 species, self-shielding, and optically thick compoents between $10$ and $10^{10}$ K. 
  \end{tabularx}
  \caption{A comparison of the various models used within both the \hagn{} and
    \fire{} simulation projects. Note that the black hole phyiscs considered in
    this table is \emph{only} presented in \citet{angles-alcazar_black_2017},
    and not in the main \fire{} paper \citep{hopkins_fire-2_2017} as the
    original set of \fire{} runs contains \emph{no black hole physics
    whatsoever}.}
  \label{tab:comparison}
\end{table}
\begin{multicols}{2}
