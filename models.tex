A quick comparison of relevant models used in \citet{dubois_black_2015}
(the \hagn{} paper, henceforth referred to as D15) and
\citet{angles-alcazar_black_2017} (the relevant \fire{} paper, henceforth
referred to as A17) is presented in Table \ref{tab:comparison}. It is important
to note that in the main \fire{} paper \citep{hopkins_fire-2_2017} and 
calculations there is no black hole physics implemented; only in A17 are
the effects of black holes considered.

Both A17 and D15 use the `zoom-in' technique; initially a large volume (in the
case of A17 this is a cube of side length 50 Mpc/$h$) is simulated using a
dark-matter only `N-body' code and a halo of Milky-Way mass at $z=0$ is
selected. This is then re-simulated at high resolution with hydrodynamics
`turned on' \citep[see][for details of the initial
simulations]{kaviraj_horizon-agn_2017, wetzel_reconciling_2016}. 

As well as the sub-grid physics models implemented in both of the simulation
suites, there are some major differences in the algorithms chosen to calculate
hydrodynamical and gravitational forces between two projects. Below these
differences are considered individually.

\subsection{Hydrodynamics}

\hagn{} and \fire{} choose fundamentally different paradigms for their
hydrodynamical calculations. \fire{} uses the `GIZMO' code in MFM mode \citep[a
meshless finite mass method, see][]{hopkins_gizmo:_2015}, whereas \hagn{} uses
the `RAMSES' code, using an AMR \citep[adaptive mesh refinment,
see][]{teyssier_cosmological_2002} technique. The difference between these two
methods is that the MFM code uses \emph{particles} to represent the fluids in
the simulations, and the AMR code uses \emph{cells} defined on a grid
structure.

The benefit to using a particle-based simulation code is that it is
automatically adaptive; regions that are denser are inherrently better resolved
as there is a higher density of particles (resolution elements). An AMR code,
on the other hand, requires a refinment algorithm which refines the mesh
typically when a cell breaches some density threshold. Historically, AMR codes
have been prized for their more accurate (typically second-order)
hydrodynamical calculations, but the GIZMO code takes advantage of a meshless
structure \citep[as opposed to the usual smoothed particle hydrodynamics used
for particle-based simulations, see e.g.][]{springel_cosmological_2005} to
solve equations to an equally high order when paired with an appropriate
Riemann solver \citep{hopkins_gizmo:_2015, hopkins_new_2017}.

Along with the underlying methods used to calculate hydrodynamical
acclerations, there is also the question of resolution. In A17, particles of
mass $m_b = 3.3\times10^4$ M$_\odot$ are used to represent the gas elements,
leading to the galaxy disk being resolved by around $10^{7.5}$ elements. They
also impose a minimal softening (force resolution) of $0.7$ pc. In contrast,
the simulations in D15 have a much lower resolution; their minimum grid size is
set to be $8.7$ pc; approximately ten times lower resolution than A17. The grid
is refined when there are more than 8 dark matter particles within a given cell
or a cell has a density 8 times greater than the intial dark matter density.
It is worth noting that these two numbers are not directly comparable; the 
chosen softening length does not necessarily correspond to a specific
`resolution scale'. Both simulation suites use dark matter particles to resolve
their halos within a factor of two of each other, suggesting the true
`resolution' of both of the suites is comparable.

\subsection{Gravity}

The two simulation suites also differ in their gravity calculation algorithm.
As with the hydrodynamics, the \hagn{} code uses an adaptive mesh to calculate
gravitation forces \citep[a `particle-mesh' method, based
on][]{kravtsov_adaptive_1997}. This has the advantage of calculating long-range
gravitational forces very accurately, but under certain circumstances can cause
errors in local force calculations. The \fire{} code uses the tree algorithm
present in \citet{springel_cosmological_2005}, which trades off long-range
accuracy for a more accurate local force calculation. With this said, it is
unlikely that the details of the algorithm used to calculate the gravitational
forces will have a marked effect on the results of a simulation; the resolution
of these methods is considerably more important. Both of the simulation suites
resolve their dark matter halos using a similar number of particles and to a
similar force accuracy.

% Change the distance between the lines to make it more readable
\renewcommand{\arraystretch}{2}

\newgeometry{margin=2cm}
\begin{table*}
  \centering
    \begin{tabularx}{\textwidth}{cXX}
    Model & \hagn & \fire \\
    \hline
        Star Formation & Schmidt Law \citep{kennicutt_star_1989}: $\dot{\rho}_* = \epsilon_* \rho / t_{\rm ff} \propto \rho^{3/2}$ at high densities ($\rho_0 > 250$ $n_H$ cm$^{-3}$), where $\dot{\rho}_*$ is the local star formation rate, $\epsilon_*$ is an efficiency factor, $\rho$ is the local gas density, and $t_{\rm ff}$ is the free-fall time. Resolution dependent. & Contains resolved molecular clouds. Star formation (SF) is activated when gas is self-gravitating, self-shielded, and has a density $\rho_0 > 1000$ $n_H$ cm$^{-3}$. SF in these regions follows a Schmidt Law: $\dot{\rho}_* = \rho_{mol} / t_{\rm ff}$ where $\rho_{mol}$ is the density of molecular gas. \\
        Stellar Feedback & Gas pressure is enhanced to follow $T = T_0 (\rho/\rho_0)$ in the cells surrounding the SF event to ensure a well-resolved \citet{jeans_stability_1902} length within four cells. & Stellar populations (SP) are individually evolved (using \textsc{Starburst99}) and use local radiative transfer (sing a Sobolev approximation to estimate the column integrated to infinity) based on the SP properties. Winds are also extracted directly from the SP and deposited locally. \\
        Supernovae & Supernovae energy of $e_{SN} = 10^{50}$ erg M$_{\odot}^{-1}$ is injected to neighboring cells (up to $30^3$ cells) using the Sedov-Taylor exact solution including mass, momentum and energy \citep[see][for more details]{dubois_onset_2008}.& A voronoi-like mesh is constructed around the star particle, and energy, momentum, and mass are deposited in a weighted manner on the faces using the exact Sedov-Taylor solution \citep[see][]{hopkins_how_2017}.\\
        Black Hole Growth & Uses a \citet{bondi_spherically_1952} solution such that $\dot{M}_{BH} \propto M_{BH}^2 \rho/(c^2 + u^2)^{3/2}$, and includes terms for black hole spin to cap accretion at the Eddington limit (see §\ref{sec:bhs}). & Uses the \citet{hopkins_analytic_2011} `Torque' formalism such that $\dot{M}_{BH} \propto f_d^{3/2} M_d R_0^{-3/2} M_{BH}^{1/6}$. The Eddington limit can be exceeded by 10 times in this model (see §\ref{sec:bhs}). \\
    Black Hole Dynamics & A drag force is added to emulate dynamical friction. & The black hole is given a dynamical mass of $M_{dyn} = 60M_{BH}$ initially. This limit is removed once $M_{BH} = M_{dyn}$. \\
    AGN Feedback & Two modes: radio ($\dot{M}_{BH}/\dot{M}_{edd} < 0.01$) where jets are ejected at $v = 10^{4}$ km s$^{-1}$, and quasar, where energy is deposited isotropically to neighboring cells such that $\dot{E}_{AGN} \propto \dot{M}_{BH}$ is calibrated to reproduce the $M_{BH}$-$M_b$ and $M_{BH}$-$\sigma_b$ relations. & None. \\
    Cooling & H, He and metals along with a UV background after reionization at $z=10$ is included in a \citet{sutherland_cooling_1993} model between $10^4$ and $10^{8.5}$ K. & A \citet{hopkins_galaxies_2014} model is used with 11 species, self-shielding, and optically thick components between $10$ and $10^{10}$ K. 
  \end{tabularx}
  \caption{A comparison of the various models used within both the \hagn{} and
    \fire{} simulation projects. Note that the black hole phyiscs considered in
    this table is \emph{only} presented in \citet{angles-alcazar_black_2017},
    and not in the main \fire{} paper \citep{hopkins_fire-2_2017} as the
    original set of \fire{} runs contains \emph{no black hole physics
    whatsoever}. Missing variable definitions can be found in the text.}
  \label{tab:comparison}
\end{table*}
\restoregeometry

