In this report, two simulation suites have been considered. The two suites use
markedly different numerical techniques to simulate isolated disk galaxies
that were extracted from cosmological simulations. A17 and D15 also use
black hole accretion rates that scale differently with local density, black
hole mass, and other properties; despite this the accretion histories for
black holes in both the \fire{} and \hagn{} simulations follow very similar
patterns, along with the observational evidence backing up their claims.
This, along with other evidence that there is not space to discuss,
lead both papers to the conclusion that efficient stellar and supernovae
feedback quench the black holes until some critical density threshold is
reached. Below the characteristic mass scale of a few $10^9$ to $10^{10}$
M$_\odot$, the bulge does not have enough mass to effectively cool gas rapidly
enough before it is destroyed by star formation and supernovae events to
feed the black hole.

Both papers include the use of similar sub-grid physics specifications, even if
they are not always consistent on the details (see Table \ref{tab:comparison}).
They both include star formation, stellar feedback, supernovae, black holes
and their dynamics, and a complex gas cooling prescription.

It is concerning that in none of the original simulation suite of 40 zoom-in
simulations from the \fire{} project there is no implementation of black holes;
it can be argued that due to the low masses of the halos ($M_{\rm halo} < 2
\times 10^{12}$ M$_\odot$) AGN would have little effect \citep[see e.g.][and
citations]{bower_breaking_2006} however the focus of the \fire{} project is to
implement as much \emph{realistic} physics as possible. It is notable that they
evolve individual stellar popualtions for each individual star particle, and
yet do not incude any black hole prescription at all; even in A17 they do not
include any AGN feedback, and just consider the feeding rate of the black hole.

Another concerning aspect of both A17 and D15 is that both papers only consider
a low number of halos (4 and 1 respectively) of approximately Milky-Way mass at
$z=0$. Whilst they do run numerical convergence tests (in which A17 performs
better than D15; see §\ref{sec:bhs}) this prevents any statistical analysis of
the data; both papers conclude that the feeding rate is dependent on the
stellar feedback (by running with and without stellar feedback `turned on')
however this could equally be due to some particular property of the initial
conditions considered. The fact that both papers come to the same conclusion is
promising, but a more statistically motivated study would have more merit.

