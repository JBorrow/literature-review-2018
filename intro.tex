Cosmological simulations play an important role in current astronomical
research; they are used to develop mock catalogues for surveys, test theories
such as modified gravity, and allow for predictions to be made about the nature
of the Universe \citep{smith_lightcone_2017, arnold_zoomed_2016,
schaye_eagle_2015}. Historically, only dark matter was simulated in an `N-body'
fashion; the effects of baryons were added later \citep{lemson_halo_2006,
lacey_unified_2016}. As computing power has increased over time, the focus has
changed to a full hydrodynamical description of the Unvierse. Simulations
usually vary three main components: the hydrodynamical scheme used to calculate
forces between gas particles, the gravitational scheme used to calculate the
long-range force of gravity, and (most importantly) the sub-grid model used to
implement star formation, stellar feedback, along with many other kinds of
`microphysics' \citep[see e.g.][]{vogelsberger_introducing_2014,
schaye_eagle_2015, pillepich_simulating_2018}.

Whilst the community has converged to two main hydrodynamical schemes and one
gravity-calculation algorithm, there still remains major differences in
sub-grid modelling between simulation suites. There is much discussion about
which physics is relevant to the key problem of galaxy formation, but there
are three main components: cooling, stellar feedback, and Active Galactic
Nuclei (AGN) feedback.

Sub-grid modelling is required to reproduce the galaxy luminosity function,
the galaxy mass function, the cosmic star formation history, among many
other key properties.

In this report, two different simulations that use different hydrodynamical
schemes and vastly different sub-grid models are compared; the \hagn{}
simulations \citep{dubois_dancing_2014}, and the \fire{} simulations
\citep{hopkins_fire-2_2017}, with a particular focus on their implementations
of black-hole physics. In §\ref{sec:models} their basic components are
compared; in §\ref{sec:bhs} their treatments of black holes are compared; and
in §\ref{sec:discussion} an overall comparison of the two simulations is given. 

